\documentclass[11pt]{amsart}

\usepackage[english]{babel}
\usepackage {amsmath} 
\usepackage{amssymb}
\usepackage{amsfonts}
\usepackage{amsthm}
\usepackage{graphicx}
\usepackage[dvipsnames]{xcolor}
\usepackage{lipsum}
\usepackage[colorlinks,linkcolor=blue, citecolor=blue,urlcolor=blue]{hyperref}
\usepackage[backend=bibtex]{biblatex}
\usepackage[colorinlistoftodos]{todonotes}

\newcommand{\info}[1]{\todo[linecolor=OliveGreen,backgroundcolor=OliveGreen!25,bordercolor=OliveGreen]{#1}}
\newcommand{\unsure}[1]{\todo[linecolor=red,backgroundcolor=red!25,bordercolor=red]{#1}}
\newcommand{\change}[1]{\todo[linecolor=Plum,backgroundcolor=Plum!25,bordercolor=Plum]{#1}}

%% Environments for theorems, etc.. 
\theoremstyle{theorem} % set the style for the following theorems
\newtheorem{thm}{Theorem}[section] %\newtheorem{name}{display-text}[numbered-within]
\newtheorem{lem}[thm]{Lemma} %\newtheorem{name}[numbered-like]{display-text}
\newtheorem{cor}[thm]{Corollary}
\newtheorem{prop}[thm]{Proposition}
\newtheorem{alg}[thm]{Algorithm}
\theoremstyle{definition}       
\newtheorem{defn}[thm]{Definition}
\newtheorem{conj}[thm]{Conjecture}
\theoremstyle{example}                     
\newtheorem{prob}[thm]{Problem}
\theoremstyle{remark}                       
\newtheorem{exmp}[thm]{Example}
\newtheorem{rem}[thm]{Remark}
\newtheorem{claim}[thm]{Claim}  
\renewcommand{\theclaim}{}

\numberwithin{equation}{section}

\newcommand{\R}{\mathbb{R}}
\newcommand{\Q}{\mathbb{Q}}
\newcommand{\N}{\mathbb{N}}
\newcommand{\Z}{\mathbb{Z}}
\DeclareMathOperator{\rank}{rank}
\DeclareMathOperator{\dimension}{dim}
\DeclareMathOperator*{\supp}{supp}
\DeclareMathOperator{\spn}{span}


\addbibresource{wavelet.bib}

\title{Introduction to Wavelets in Image Processing}
\author{Jason Ngo}
\date{2019-04-03}

\begin{document}
\maketitle

\section{Introduction}
Since the start of the 20th century, we have seen rapid development in the theory and applications of wavelets. As a mathematical tool, wavelets can be used to extract information from different kinds of data such as audio signals and images. As an attempt to explore wavelets at an introductory level, this paper will examine the first wavelet developed, called the Haar Wavelet and discuss its applications in image processing.

\vspace{8pt}
First, recall that \emph{$ L^2(\R) $} is a vector space of square integrable functions, taken with the inner product
\[ \langle f,g \rangle = \int_{\R} f(x) g(x)  dx. \]

\begin{defn} \label{haar}
	The \emph{Haar function} is the function $ \varPsi = \chi_{[0,0.5)} - \chi_{[0.5,1)} $. The \emph{Haar system} is the family
	\[ \{ \varPsi_{j,k}(x) = 2^{j/2} \varPsi (2^j x-k), \qquad j,k \in \Z \}. \]
\end{defn}

	Note that each term in the Haar system is constructed by translating and/or dilating the original $ \varPsi $ (see Figure \ref{fig:haarsystem}). By this construction, since $ \int_{\R} \varPsi(x)dx = 0 $, it follows that $ \int_{\R} \varPsi_{j,k}(x)dx = 0 $ for $ j,k \in \Z $.
	Furthermore, it is important to calculate the width and height for each $ \varPsi_{j,k} $. We have $ \varPsi(2^j - k) = 1 $ when $ 0 \leq 2^j - k < \frac{1}{2} $, or equivalently, $ \frac{k}{2^j} \leq x < \frac{k+\frac{1}{2}}{2^j} $. Therefore, in this interval, $ \varPsi_{j,k} = 2^j $. Similar calculations yield:
	\begin{equation} \label{eq:height}
		\varPsi_{j,k} = 
		\begin{cases}
		2^{j/2} &\text{if}\ \frac{k}{2^j} \leq x < \frac{k+\frac{1}{2}}{2^j}, \\
		-2^{j/2} &\text{if}\ \frac{k+\frac{1}{2}}{2^j} \leq x < \frac{k+1}{2^j}, \\
		0 &\text{otherwise}.
		\end{cases}
	\end{equation}

\begin{figure}[h]
	\centering
	\includegraphics[width=0.7\linewidth]{img/haar_system}
	\caption[Elements of the Haar system]{Elements of the Haar system}
	\label{fig:haarsystem}
\end{figure}
	
Now, we will prove that the Haar function $ \varPsi $ satisfies Definition \ref{def:wavelet} of an \emph{orthonormal wavelet}.

\begin{defn}[Orthonormal Wavelet, {\cite[303]{pinsky}}] \label{def:wavelet}
	If $ \varPsi \in L^2(\R) $, $ \varPsi_{j,k}(x) = 2^{j/2} \varPsi (2^j x-k) $, and the set $ \{ \varPsi_{j,k}: j,k \in \Z \} $ is an orthonormal basis for $ L^2(\R) $, then $ \varPsi $ is called an \emph{orthonormal wavelet}.
\end{defn}

To prove the Haar system is an \textit{orthonormal basis}, we have to show:
	\begin{enumerate}
		\item The Haar system $ \{ \varPsi_{j,k} \} $ is an orthonormal set, i.e. $ \| \varPsi_{j,k} \| = 1 $ for all $ j,k \in \Z $ and $ \langle \varPsi_{j,k}, \varPsi_{j',k'} \rangle = 0 $ for all $ (j,k) \neq (j',k') $.
		
		\item The span of Haar system, denoted $ \spn\{\varPsi_{j,k}\} $, is dense in $ L^2(\R) $.
		\unsure{Does it mean if a basis spans a dense subspace, it will span other subspaces? Since there are so many dense subspaces of $ L^2(\R) $, which one do we choose among $ C_{00}, C_0, C_c $?}
	\end{enumerate}

In Section \ref{section:orthonormality}, we will first prove that the Haar system is orthonormal. Section \ref{section:span} will show that it spans $ L^2(\R) $, thereby completing our proof that the Haar function is indeed a wavelet. Finally, we will discuss the application of Haar wavelet in image processing in Section \ref{section:application}.

\section{Orthonormality} \label{section:orthonormality}
\begin{lem}[{\cite[409]{davidson}}] \label{lem:orthonormal}
	The Haar system is an orthonormal set in $ L^2(\R) $.
	
	\begin{proof}
		First, it is easy to see that $ \| \varPsi \| = 1 $. Let us compute:
		\begin{align*}
		\| \varPsi_{j,k} \|^2 = \int_{\R} \left| 2^{j/2} \varPsi(2^{j} x - k) \right|^2 dx 
		&=   \int_{\R} 2^j \left| \varPsi(2^{j} x - k) \right|^2 dx \\
		&= \int_{\R} \left| \varPsi(y) \right|^2 dy
		= \| \varPsi \|^2,
		\end{align*}
		by change of variable $ y = 2^{j}x-k $ in the integral. Therefore, all the elements in the Haar system has norm 1.
		
		Now, we want to show the orthogonality. Consider $ \varPsi_{j,k} $ and $ \varPsi_{j,k'} $, for $ k \neq k' $. Since these two elements have disjoint support\footnote{The \emph{support} of a function, denoted $ \supp $, is the subset of the domain containing those elements which are not mapped to zero.}, their inner product is 0. If $ j < j' $, then either $ \varPsi_{j,k} $ and $ \varPsi_{j',k'} $ have disjoint supports (when $ k \neq k' $), or supp $ \varPsi_{j',k'} $ is contained in an interval on which $ \varPsi_{j,k} $ is constant to (when $ k = k' $). For the latter case, we have:
		\[ \langle \varPsi_{j,k}, \varPsi_{j',k'} \rangle =  \int_{\R} 2^{j/2} \cdot \varPsi_{j',k'} dx = 2^{j/2} \int_{\R} \varPsi_{j',k'} dx = 0. \]
	
		Hence, the set is orthonormal.
	\end{proof}
\end{lem}

\begin{rem}
	Haar function is a dyadic function, meaning that the dilations are taken to be powers of 2. Here, the factor $ 2^{j/2} $ is called the normalization factor, which is there so that the dilated and translated Haar function has norm 1. Although powers of 2 are a common choice, they are certainly not the most general one\footnote{See examples of non-dyadic wavelets and a discussion why non-dyadic wavelets are more appropriate for statistical data analysis in \cite{pollock}.}.
\end{rem}

Now that we have proved the Haar system is orthonormal, in the following sections, we continue proving that the Haar function satisfies Definition \ref{def:wavelet} of \emph{orthonormal wavelet}.

\section{Haar Wavelet Basis for $ L^2(\R) $} \label{section:span}
\subsection{Inner Product Expansion}
Motivated by \cite[3]{bell} and \cite[516]{davidson}, this subsection will set up the inner product expansion $ P_nf $ below, which will act as the foundation to prove that any function in $ L^2(\R) $ is in the span of $ \{ \varPsi_{j,k} \} $.\unsure{be careful with the definition}

\vspace{8pt}
Set $ \phi(x) = \chi_{[0,1)}$. For $ n \in \Z $, we define
\[ K_n (x,y) = 2^n \sum_{k \in \Z} \phi(2^n x - k) \phi(2^n y - k). \]

Note that $ K_n $ is either equal to $ 2^n $ (when there is some $ k \in \Z $ s.t. $ 2^n x-k $ and $ 2^ny - k $ lie in $ [0,1) $) or equal to 0 (otherwise). Equivalently, if $ K_n = 2^n $, there is some $ k $ s.t.
\[ x,y \in \left[ \frac{k}{2^n}, \frac{k+1}{2^n} \right) = I_{n,k}. \]
Here, note that $ K_n(x,y) $ is constant (equal to $ 2^n $) on each dyadic interval of length $ 2^{-n} $. Furthermore, for any given $ n \in \Z $, the set $ \{ I_{n,k} \} $ is disjoint.

We define
\[ P_n f(x) = \int_{\R} K_n(x,y) f(y) dy. \]
If $ x \in \R $ then there is a unique $ k_x \in \Z $ with $ x \in I_{n, k_x} $ and
\[ P_n f(x) = 2^n \int_{I_{n,k_x}} f(y) dy. \]

\begin{lem}[{\cite[293]{pinsky}}]
	If $ n \in \Z $, then
	\[ K_{n+1} - K_n = \sum_{k \in \Z} \varPsi_{n,k}(x) \varPsi_{n,k}(y),\qquad x,y \in \R. \]
	
	\begin{proof}
		By Equation \ref{eq:height}, $ \varPsi_{n,k}(x) $ is equal to $ 2^{n/2} $ when $ x \in I_{n+1,2k} = \left[\frac{2k}{2^{n+1}}, \frac{2k+1}{2^{n+1}}\right) $ and is equal to $ -2^{n/2} $ when $ x \in I_{n+1,2k+1}$. Therefore, it follows that: \change{I don't know how to phrase this better.}
		\[
		\varPsi_{n,k}(x) \varPsi_{n,k} (y) =
		\begin{cases}
		2^n &\text{if}\ x,y \in I_{n+1,2k}\ \text{or}\ x,y \in I_{n+1,2k+1}, \\
		-2^n &\text{if one is in }\ I_{n+1,2k}\ \text{and the other is in}\ I_{n+1,2k+1}, \\
		0 &\text{otherwise}.
		\end{cases}
		\]
		
		Before calculating $ K_{n+1} - K_n $, note that:
		\[
		I_{n+1,2k} \cup I_{n+1,2k+1} = \left[ \frac{2k}{2^{n+1}}, \frac{2k+2}{2^{n+1}} \right) = I_{n,k}. 
		\]
		
		Therefore, if $ x,y $ is in either $ I_{n+1,2k} $ or $ I_{n+1,2k+1} $, then $ x,y \in I_{n,k} $ and $ K_n(x,y) = 2^n $. Hence, for this case, we have:
		\[ K_{n+1}(x,y) - K_n(x,y) = 2^{n+1} - 2^n = 2^n. \]
		
		By similar argument, if either $ x $ or $ y $ is in $  I_{n+1,2k} $ and the other is in $ I_{n+1,2k+1} $, then
		\[ K_{n+1}(x,y) - K_n(x,y) = 0 - 2^n = -2^n. \]
		
		Otherwise, everything is 0.
		
		Hence, it follows that $ K_{n+1} - K_n = \sum_{k \in \Z} \varPsi_{n,k}(x) \varPsi_{n,k}(y) $.
	\end{proof}
\end{lem}

\begin{rem}
	One important thing that this lemme does is to get rid of the scaling function and write integral kernel as the sum of the product of two functions from the Haar system.
	
	Furthermore, from this lemma, we also get
	\begin{align*}
		(P_{n+1} - P_n) f(x) &= \int_{\R} K_{n+1}(x,y)f(y)dy - \int_{\R} K_{n}(x,y)f(y)dy \\
		&= \int_{\R} \sum_{k \in \Z} \varPsi_{n,k}(x) \varPsi_{n,k}(y) f(y) dy \\
		&= \sum_{k \in \Z} \varPsi_{n,k}(x) \int_{\R} \varPsi_{n,k}(y) f(y) dy \\
		&= \sum_{k \in \Z} \langle f, \varPsi_{n,k} \rangle \varPsi_{n,k}(x).
	\end{align*}
	Damn, we just wrote something dependent on the scaling function entirely on the wavelet series.
	
	We call $ \langle \rangle $ the Haar coefficients. This is why this subsection is called the Haar inner product expansion.
\end{rem}

\subsection{Span of Haar System}
\begin{lem}[Convergence of Inner Product Expansion, {\cite[7]{bell}, \cite[517]{davidson}}]
	If $ f \in C_c(\R) $, then $ P_nf \to f $ in the uniform nom as well as in the $ L^2 $ norm.
	
	\info{draw a picture of $ P_nf $}
	\begin{proof}
		Suppose $ f \in C_c(\R) $ and $ \supp(f) \subset [-2^M, 2^M] $ for $ M \geq 0 $. Since $ f $ is continuous on a compact set, it is uniformly continuous.
		
		Given $ \epsilon > 0 $. By uniform continuity, there exists $ \delta > 0 $ such that if $ |x-y| < \delta $, then $ |f(x) - f(y)| < \epsilon $.
		
		Now, choose $ N $ such that $ 2^{-N} < \delta $; if $ n > N $, then we have:
		\begin{align*}
			|P_nf(x) - f(x)| &= \left| 2^n \int_{I_{n,k_x}} f(y)dy - f(x) \right| \\
			&=  \left| 2^n \int_{I_{n,k_x}} f(y)dy - 2^n \int_{I_{n,k_x}} f(x) dy \right| \\
			&\leq 2^n \int_{I_{n,k_x}} |f(y) - f(x)| dy\\
			&< 2^n \int_{I_{n,k_x}} \epsilon dy = \epsilon.
		\end{align*}
		Therefore, $ P_nf $ converges to $ f $ uniformly. Furthermore,
		\begin{align*}
			\| P_nf - f \|_2^2 &= \int_{\R} |P_nf(x) - f(x)|^2dx \\
			&= \int_{-2^M}^{2^M} |P_nf(x) - f(x)|^2dx \\
			&\leq \int_{-2^M}^{2^M} \|P_nf - f\|_\infty^2dx = 2 \cdot 2^M \| P_n - f \|_\infty^2.
		\end{align*}
		Since the right hand side converges to 0, by limit comparison test, it follows that $ \| P_nf-f \|_2^2 \to 0 $ as $ n \to \infty $.
		
		Hence, $ P_n \to f $ both in the uniform norm and in the $ L^2 $ norm.
	\end{proof}
\end{lem}

\info{what if the continuity hypothesis is violated mildly?}

\info{draw a picture of $ P_nf $ and $ f $ with different $ n $}

\info{what is the definition of span}
\begin{thm}[{\cite[411]{davidson}}] \label{span}
	The Haar system spans all of $ L^2(\R) $.
	
	\begin{proof}
		Let $ f $ be arbitrary in $ \in L^2(\R) $. Given $ \epsilon > 0 $.
		
		Since $ C_c(\R) $ is dense in $ L^2(\R) $\footnote{The proof for this claim can be found in \cite[326]{farrell}}, there exists $ g \in C_c(\R) $ s.t. $ |f - g| <  $
		
		\begin{align*}
			\|P_nf - f\|_2 &\leq \|P_nf - P_ng\|_2 + \|P_ng - g\|_2 + \|g - f\|_2 \\
			&\leq 
		\end{align*}
	\end{proof}
\end{thm}

Therefore, the Haar function is indeed an orthonormal wavelet. We will now refer to the Haar function as the \emph{Haar wavelet}.



\begin{exmp}
	Consider $ f(x) = e^{-x} \sin 2\pi x $ for $ x \in (0,1) $.
	
	See Figure \ref{fig:approximations} 
\end{exmp}

\section{Application to Image Processing} \label{section:application}
 When compressing images, we want to discard the least significant details, keeping the original picture largely intact. Fortunately, wavelets can isolate and decompose a signal into low frequency part and high frequency part.
 Briefly discuss FBI Fingerprint Image Compression if there is space:
 \begin{quote}
 	 Wavelet compression methods do not require dividing the image into smaller blocks because the desired localization properties are naturally built into the wavelet system.\cite{frazier}
 \end{quote}

\section{Scratch}
\begin{exmp}
	Let $ f = \chi_{[0,1} $. Prove that the Haar series is not convergent in
\end{exmp}

\printbibliography

\end{document}

