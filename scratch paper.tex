\documentclass[2pt]{amsart}

\usepackage[english]{babel}
\usepackage {amsmath} 
\usepackage{amssymb}
\usepackage{amsfonts}
\usepackage{amsthm}
\usepackage{graphicx}
\usepackage[colorlinks,linkcolor=blue, citecolor=blue,urlcolor=blue]{hyperref}
\usepackage[backend=bibtex]{biblatex}

%% Environments for theorems, etc.. 
\theoremstyle{theorem} % set the style for the following theorems
\newtheorem{thm}{Theorem}[section] %\newtheorem{name}{display-text}[numbered-within]
\newtheorem{lem}[thm]{Lemma} %\newtheorem{name}[numbered-like]{display-text}
\newtheorem{cor}[thm]{Corollary}
\newtheorem{prop}[thm]{Proposition}
\newtheorem{alg}[thm]{Algorithm}
\theoremstyle{definition}       
\newtheorem{defn}[thm]{Definition}
\newtheorem{conj}[thm]{Conjecture}
\theoremstyle{example}                     
\newtheorem{prob}[thm]{Problem}
\theoremstyle{remark}                       
\newtheorem{exmp}[thm]{Example}
\newtheorem{rem}[thm]{Remark}
\newtheorem{claim}[thm]{Claim}  
\renewcommand{\theclaim}{}

\numberwithin{equation}{section}

\newcommand{\R}{\mathbb{R}}
\newcommand{\Q}{\mathbb{Q}}
\newcommand{\N}{\mathbb{N}}
\newcommand{\Z}{\mathbb{Z}}
\DeclareMathOperator{\rank}{rank}
\DeclareMathOperator{\dimension}{dim}
\DeclareMathOperator*{\supp}{supp}

\addbibresource{wavelet.bib}


\author{Jason Ngo}
\begin{document}
	
\begin{quote}
	The child wavelets $ \varPsi_{j,k} $ are simply dilated and translated versions of the mother wavelet multiplied with the normalization factor 2j/2. The normalization factor is there so that the dilated and translated Haar function satisfies property (2) in the wavelet	definition.
\end{quote}

Observe that all $ \varPsi_{j,k} $ are generated by shifting and scaling\footnote{The dilations here are taken to be powers of 2.} the wavelet $ \varPsi $ and that each $ \varPsi_{j,k} $ is normalized so that $ \| \varPsi_{j,k}\|_2 = \|\varPsi\|_2 = 1 $ for all $ j,k \in \Z $.

---

i.e., the intervals on which they are non-zero are different; therefore, when the two functions multiply together, the result must be zero (therefore the 2-norm is 0). Thus, they are orthogonal. 
\begin{quote}
	Now, if $ j < j' $, then $ \varphi $ and $ \varPsi_{j,k} $ are constant on the support of $ \varPsi_{j',k'} $. Since $ \int_0^1 \varPsi_{j,k}(x) = 0 $ for all $ j $ and $ k $, it now follos that these functions are pairwise orthogonal.
\end{quote}

\section{Useful quotes}
\begin{quote}
	Wavelets are local functions that enable us to cut up data into different layers of frequency. A wavelet basis is formed by translating and dilating a small wave, making it
	possible to analyze data at different scales. Although wavelet analysis is promising, it
	has not entered mainstream study of economic phenomena. The aim of this thesis is to
	give an intuitive theoretical understanding of wavelets, and describe how they can be
	used in time series analysis. Applications for economic time series are presented, as well
	as some thoughts of how the field of economics will progress due to wavelet analysis.
\end{quote}


\section{Annotated Bibliography}
\begin{enumerate}
	\item \fullcite{davidson_real_2002}
	
	\smallskip
	This textbook is my primary source for proofs of orthonormal basis. Though terse, the book introduces the lemmas and proofs in a logical way, starting with proving the Haar system is orthonormal and then expand it to the basis for $L^2(\R)$.
	
	\item \fullcite{Frazier_1999}
	
	\smallskip
	This textbook is very introductory, including a lot of examples and step-by-step proof for the properties of wavelets. Furthermore, it also has a nice description of the FBI Fingerprint Compression application using Haar Wavelets Analysis.	
	
	\item \fullcite{Gomes_Velho_2015}
	
	\smallskip
	Although this textbook is not very proof-heavy, it brings up a lot of cool wavelet examples that pertain to image compression, one of which being the Blur Derivative that I can talk about in my first draft.
\end{enumerate}	
	
\end{document}