\documentclass[11pt]{amsart}

\usepackage[backend=bibtex]{biblatex}
\addbibresource{sample.bib}

\usepackage[colorlinks,urlcolor=blue]{hyperref}

\title{Using biblatex for an annotated bibliography}
\author{David Lippel}
\date{2019-02-04}

\begin{document}

\begin{abstract}       
In this note, I illustrate one way to produce an annotated bibliography with the biblatex package.
\end{abstract} 

\maketitle




\section{Introduction}

The \LaTeX\ template document that is posted on Moodle has general information about biblatex. 
For completeness, the current document does repeat some of the information from the template, 
but if you don't yet know how BibTeX/biblatex works in general, then you will need to consult the template file.

\section{biblatex essentials}

Here are the  essential features of the current document that allow biblatex to function.
\begin{itemize}
\item At the beginning of the source file, we have the lines
\begin{verbatim}
\usepackage[backend=bibtex]{biblatex}
\addbibresource{sample.bib}
\end{verbatim}
\item In the same directory as the source file, there is a file called \texttt{sample.bib} which contains the bibliographic records. 
The name of this file is referenced in the \verb+\addbibresource+ command above. 
If you use a different name for your \texttt{.bib} file, don't forget to update the %\verb+\addbibresource+ 
command.
\end{itemize}

\section{Typesetting references with annotations}

In a typical document, we would use the biblatex command \verb+\printbibliography+  
at the very end of the source file in order to print the whole reference list. 
However, in an annotated bibliography, we want individual biblographic entries to be interspersed with text. 
To typeset an individual bibliographic entry, we can use the \verb+\fullcite{...}+ command. 
Like the \verb+\cite{...}+ command, the argument for \verb+\fullcite+ is the citation key from the \texttt{.bib} file.
(Recall: the citation key is how you reference a bibliographic item, and it is the first field in a BibTeX record. 
If you get your BibTex data from an online resource,  the default citation key maybe be obscure;
you can edit your \texttt{.bib} file to make the citation keys memorable.)

Below, I have provided a sample annotated bibliography; look at the source code to see how the \verb+\fullcite+ command works.

\subsection{My rather useless annotated bibliography}\footnote{Yours should be more thoughtful!}

\begin{enumerate}
\item \fullcite{MR1531724}

\smallskip
When Carl Allendoerfer wrote this article, he was retiring as the President of the Mathematical Association of American and was a faculty member at the University of Washington; previously, he had been a professor at Haverford College. (You can see Allendoerfer's photo in H215, the computer room off the math lounge.)

\item \fullcite{tabu}

\smallskip
This online resource is only for seriously fussy \LaTeX ers.

\item \fullcite{Di1988}

\smallskip
This looks interesting, but I haven't read it yet.
\end{enumerate}




\section{Workflow}\label{workflow}

Typesetting a \LaTeX\ source file that uses biblatex is slightly more complicated than typesetting a plain vanilla \LaTeX\ document. If you are using a \LaTeX\ system installed on a PC or Mac, you should consult the template file for the extra steps. 
A nice alternative is to use the online \LaTeX\ service \href{https://www.overleaf.com/}{Overleaf}. 
On Overleaf, BibTeX processing is done automatically, without any extra steps. 
An account on Overleaf is free. (However, synchronization with Dropbox is a premium feature. If you use the free version and thus don't have an automatic backup, I recommend that you export your project periodically to keep a local copy.)

\end{document}

